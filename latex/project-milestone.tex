\documentclass{article}

% if you need to pass options to natbib, use, e.g.:
%     \PassOptionsToPackage{numbers, compress}{natbib}
% before loading neurips_2021

% ready for submission
% \usepackage{neurips_2021}

% to compile a preprint version, e.g., for submission to arXiv, add add the
% [preprint] option:
\usepackage[preprint]{neurips_2021}

% to compile a camera-ready version, add the [final] option, e.g.:
%     \usepackage[final]{neurips_2021}

% to avoid loading the natbib package, add option nonatbib:
%    \usepackage[nonatbib]{neurips_2021}

\usepackage[utf8]{inputenc} % allow utf-8 input
\usepackage[T1]{fontenc}    % use 8-bit T1 fonts
\usepackage{hyperref}       % hyperlinks
\usepackage{url}            % simple URL typesetting
\usepackage{booktabs}       % professional-quality tables
\usepackage{amsfonts}       % blackboard math symbols
\usepackage{nicefrac}       % compact symbols for 1/2, etc.
\usepackage{microtype}      % microtypography
\usepackage{xcolor}         % colors

\title{CSE 151B Project Milestone Report}

% The \author macro works with any number of authors. There are two commands
% used to separate the names and addresses of multiple authors: \And and \AND.
%
% Using \And between authors leaves it to LaTeX to determine where to break the
% lines. Using \AND forces a line break at that point. So, if LaTeX puts 3 of 4
% authors names on the first line, and the last on the second line, try using
% \AND instead of \And before the third author name.

\author{%
    Alexander Friend\\
    \texttt{apfriend@ucsd.edu}
}

\begin{document}

\maketitle

\begin{abstract}
    The abstract paragraph should be indented \nicefrac{1}{2}~inch (3~picas) on
    both the left- and right-hand margins. Use 10~point type, with a vertical
    spacing (leading) of 11~points.  The word \textbf{Abstract} must be centered,
    bold, and in point size 12. Two line spaces precede the abstract. The abstract
    must be limited to one paragraph.
\end{abstract}

\section{Task Description and Exploratory Analysis}
    \subsection{Problem A [1 points]}        
        \textbf{
            Describe in your own words what the task is and why it is
            important. Define the input and output in mathematical language and formulate your
            prediction task. Refer to week 1 lectures if you need help.
        }
        
        The task for this Kaggle competition is to predict the positions of 60 individual 
        vehicles 3 seconds into the future, given an initial 2 second observation. This is 
        an important task as autonomous vehicles (AV) are being increasingly rolled out to the public,
        and are expected to become the future standard of automobile transportation. In order 
        for this future to be realized however, AV's must be able to predict future movement and positions
        of objects in their visinity with high accuracy in order for AV's to be safer than human drivers.

        Our data is split into two sets, a training set and test set, title {\fontfamily{qcr}\selectfont new\_train} and 
        {\fontfamily{qcr}\selectfont new\_val\_in}, respectively. The training dataset contains the following fields: 
        
        \begin{itemize}            
            \item {\fontfamily{qcr}\selectfont p\_in} - the (x,y) position input in the first two seconds (19 time steps)
            \item {\fontfamily{qcr}\selectfont v\_in} - the (x,y) velocity input in the first two seconds (19 time steps)
            \item {\fontfamily{qcr}\selectfont p\_out} - the (x,y) position output in the next three seconds (30 time steps)
            \item {\fontfamily{qcr}\selectfont v\_out} - the (x,y) velocity output in the next three seconds (30 time steps)
            \item {\fontfamily{qcr}\selectfont track\_id} - the track\_id for each vehicle in the output sequence (30 time steps)
            \item {\fontfamily{qcr}\selectfont scene\_idx} - the id of this scene
            \item {\fontfamily{qcr}\selectfont agent\_id} - track id for the agent in this scene
            \item {\fontfamily{qcr}\selectfont car\_mask} - boolean index for the real car, we need to align the car numbers
            \item {\fontfamily{qcr}\selectfont lane} - (x,y,z) for centerline nodes in this scene (z position is always $0$)
            \item {\fontfamily{qcr}\selectfont lane\_norm} - (x,y,z) the direction of each lane node (z position is always $0$)
        \end{itemize}

        The test set contains the same fields, except it is lacking the {\fontfamily{qcr}\selectfont p\_out} and
        {\fontfamily{qcr}\selectfont v\_out} fields, as these are the fields to be predicted. 
        
        Using these datasets, our input will be the the positions ({\fontfamily{qcr}\selectfont p\_in}), velocties 
        ({\fontfamily{qcr}\selectfont v\_in}), and ID's of $60$ other vehicles ({\fontfamily{qcr}\selectfont track\_id}) 
        in the scene, as well as information on the position of the lanes. Lane information will include the position
        of the center of the lanes (({\fontfamily{qcr}\selectfont lane})) in the scene and the direction of each lane 
        ({\fontfamily{qcr}\selectfont lane\_norm}). There can be more than one lane in a scene.
        Using all the data in this input set would result with $19$ position and velocity
        values for each vehicle over a period of $2$ seconds and $\left( k \times 3 \right)$ values for each $k$ lane position(s) 
        and direction(s). This gives us 
        $\left( 2 \times \left( 19 \times 2 \right) \times 60 \right) + \left( 2 \times 3k \right) = 4566k$ dimensions in each
        input datapoint for $k$ lanes in each scene.

        The output will be the ID of each of the $60$ other vehicles in the scene ({\fontfamily{qcr}\selectfont scene\_idx}) 
        with their respective positions ({\fontfamily{qcr}\selectfont p\_out}). The output will have $10$ records/second 
        for the $3$ second prediction period for each of the $60$ vehicles. This means that the output csv file will have 
        $\left( 10 \times 3 \times 60 \right) + 1 = 3201$,
        rows including the header, as each row contains the {\fontfamily{qcr}\selectfont scene\_idx}, and 
        {\fontfamily{qcr}\selectfont p\_out} of all the vehicles in that time frame. 
        It will therefore have $61$ columns, one column for each vehicle position ({\fontfamily{qcr}\selectfont p\_out})
        as well as a column for the corresponding scene id ({\fontfamily{qcr}\selectfont scene\_idx}).



    \subsection{Problem B [1 points]}        
        \textbf{
            Run the provided Jupyter notebook for loading the data. Provide
            exploratory analysis on the data and report your findings with texts and figures. Your
            report should answer the following questions at a minimum:
            \begin{itemize}
                \item what is the train/test data size, how many dimensions of inputs/outputs
                \item what is the distribution of input positions for all agents (hint: use histogram)
                \item what is the distribution of output positions for all agents (hint: use histogram)
                \item what is the distribution of velocity (magnitude) of all agents and the target agent
            \end{itemize}
            If you include more exploratory analysis beyond the above questions that provides in-
            sights into the data, you will receive bonus points.
        }

        The training data set contains $205,944$ training values, each of dimension
        $2 \times 60 \times 19 \times 4$. Each training value contains the input and output positions
        and velocities of each of the $60$ cars $1.9$ seconds, at a sampling rate of $10\mathrm{Hz}$.

        The validation set contains $3,200$ values, each of dimension
        $2 \times \times 60 \times 19 \times 4$. Each training value contains the input and output positions
        and velocities of each of the $60$ cars $1.9$ seconds, at a sampling rate of $10\mathrm{Hz}$.


\section{Deep Learning Model and Experiment Design}
\label{gen_inst}
    \subsection{Problem A [1 Points]}
        \textbf{
            Describe how you set up the training and testing design for deep
            learning. Answer the following questions:
            \begin{itemize}
                \item What computational platform/GPU did you use for training/testing?
                \item What is your optimizer? How did you tune your learning rate, learning rate decay,
                momentum and other parameters?
                \item How did you make multistep (30 step) prediction for each target agent?
                \item How many epoch did you use? What is your batch-size? How long does it take to
            train your model for one epoch (going through the entire training data set once)?
            \end{itemize}
            Explain why you made these design choices. Was it motivated by your past experience?
            Or was it due to the limitation from your computational platform? You are welcome to
            use screenshots or provide code snippets to explain your design.
        }

        My model will be a PyTorch \textit{nerual network (make sure to describe actual model)}, running in an Anaconda environment.
        The platform I am currently working in is my local machine running Ubuntu 20.04 with a 4-core 4-thread Intel i7-7600k CPU 
        running at 4.2 GHz, a GTX 1070 GPU with a max clock speed of 1721 MHz and 8 GB of GDDR5 memory, and 16 GB of 2400 MHz DDR4 memory.
        
        
    \subsection{Problem B [1 Points]}        
        \textbf{
            Describe the models you have tried to make predictions. You
            should always start with simple models (such as Linear Regression) and gradually in-
            crease the complexity of your model. Include pictures/sketch of your model architecture
            if that helps. You can also use mathematical equations to explain your prediction logic.
        }


\section{Experiment Results and Future Work}
\label{headings}
    \subsection{Problem A [1 points]}

        \textbf{
            Play with different designs of your model and experiments and
            report the following for your best-performing design:
            \begin{itemize}
                \item Visualize the training loss (RMSE) value over training steps (You should expect to
                see an exponential decay).
                \item Randomly sample a few training samples after the training has finished. Visualize
                the ground truth and your predictions
                \item Your current ranking on the leaderboard and your test RMSE.
            \end{itemize}
            Summarize your current experiment results. If you have tried more than one experiment
            design, compare all of them in a table/figure.
            Analyze the results and identify the lessons/issues that you have learned so far. Briefly
            discuss what you plan to do to improve the performance in the following weeks.
        }


\end{document}

